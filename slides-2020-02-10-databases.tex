% !TEX encoding = UTF-8 Unicode

\documentclass[aspectratio=169]{beamer}
\usepackage[utf8]{inputenc}

%% GRAPHICS

\usepackage{graphicx}
\graphicspath{{images/},{figures/}}
\usepackage{tikz}

%% COLORS
\definecolor{UWRed}{HTML}{C5050C}
\definecolor{TintedBG}{HTML}{EEEEFF}
\definecolor{DarkGreen}{HTML}{10A010}
\definecolor{LightGreen}{HTML}{A0FFA0}

%% SLIDE COLOR SETTINGS
\setbeamercolor{structure}{fg=UWRed}
\setbeamercolor{title page}{fg=white}
\setbeamercolor{title}{fg=white}

%% RM NAV SYMBOLS
\setbeamertemplate{navigation symbols}{}

%% FONTS
\setbeamerfont{title}{size=\huge\bfseries}

%% LOGO on slides
\logo{\begin{tikzpicture}[overlay]
	\node[anchor=north east,inner sep=0] at (0,86mm) {\includegraphics[height=10mm]{SMPH_color-flush.pdf}};
	\end{tikzpicture}}

%% CONTENT BEGINS

\title{Database Concepts}
\subtitle{BMI 773: Clinical Research Informatics}
\author{Yuriy Sverchkov}
\institute{University of Wisconsin--Madison}
\date{February 10, 2020}

\begin{document}
	
	{
		\setbeamertemplate{background canvas}{\includegraphics[width=\paperwidth]{UW-lake.png}}
		\begin{frame}[plain]
			\vskip4cm
			\titlepage
		\end{frame}
	}

	\begin{frame}{Database Concepts}
		\begin{itemize}
			\item Transactional databases vs data warehouses
			\item The relational model
			\begin{itemize}
				\item Tables, records, and columns
				\item Schemas
				\item Normal forms
				\item Keys
			\end{itemize}
			\item Learn how to design and execute SQL queries
		\end{itemize}
	\end{frame}

	\begin{frame}{Transactional databases vs data warehouses}
	\end{frame}

	\begin{frame}{}
		\centering
		\begin{tikzpicture}[
			whitenode/.style={fill=white, draw=none},
			bluenode/.style={fill=blue!20, draw=blue!80},
			greennode/.style={fill=LightGreen, draw=DarkGreen}
		]
			\matrix (table) [
			ampersand replacement=\&,
			row sep=0.1cm, column sep=0.1cm,
			nodes={minimum width=2.5cm, minimum height=0.8cm, draw=black!80, very thick, fill=black!20, font=\ttfamily}
			] {
				\node[whitenode] {first\_name}; \& \node[whitenode] (attr) {\color<2>{blue!80}{last\_name}}; \&  \node[whitenode] {gender}; \& \node[whitenode] {birth\_date};\\
				\node {Lauren}; \& \alt<2>{\node[bluenode] }{\node } {Johnson}; \& \node {F}; \& \node {1982-03-12}; \\
				\alt<3>{\node[greennode] (record)}{\node (record)} {Peter}; \& \alt<2>{\node[bluenode]}{\alt<3>{\node[greennode]}{\node}} {Jurasik}; \& \alt<3>{\node[greennode]}{\node} {M}; \&  \alt<3>{\node[greennode]}{\node} {1950-04-25};  \\
				\node {Richard}; \& \alt<2>{\node[bluenode]}{\node} {Biggs}; \& \node {M}; \& \node {1960-03-18}; \\
				\node {Lauren}; \& \alt<2>{\node[bluenode]}{\node} {Johnson}; \& \node {F}; \& \node {1974-06-01}; \\
			};
		
			\only<1->{
				\node[anchor=south east, color=purple!80, font=\bfseries, align=center] at (table.north west) {Table\\ Relation\\ Relvar}; }
			\only<2->{
				\node[anchor=south, color=blue!80, font=\bfseries, align=center] at (attr.north) {Column\\Attribute\\Field};
			}
			\only<3->{
				\node[anchor=east, color=DarkGreen, font=\bfseries, align=center] at (record.west) {Row\\Tuple\\Record};
			}
		\end{tikzpicture}
	\end{frame}

	% Do a thing on primary key/surrogate key/natural key?

	\begin{frame}{Why have more than one table?}
	
	{\tiny
		\begin{tabular}{lllllll}
			\bf First Name & \bf Last Name & \bf Birth Date \only<2->{& \bf Drug & \bf Prescription date} \only<3>{& \bf Drug 2 & \bf Prescription date 2} \\ \hline
			Lauren & Johnson & 1982-03-12 \only<2->{& Acetaminophen & 2010-01-06} \only<3>{& --- & ---} \\
			Alice & Smith & 1979-12-04 \only<2->{& Amoxicillin & 1998-01-29 } \only<3>{& Lisinopril & 2019-05-26 } \\
			\only<4->{Alice & Smith & 1979-12-04 & Lisinopril & 2019-05-26 \\}
			James & White & 1964-09-01 \only<2->{& Albuterol & 1990-12-15 } \only<3>{& --- & ---} \\
		\end{tabular}
	}
	\vfill
		\begin{itemize}
			\item Consider a database of patients \only<2->{and their prescriptions}
			\item<2-> How can we track multiple prescriptions for a patient?
				\begin{itemize}
					\item<3-> Add columns?
					\item<4-> One row per prescription?
				\end{itemize}
		\end{itemize}
	\end{frame}

	\begin{frame}
		
		{\tiny
			\begin{tabular}{llll}
				\bf Patient ID & \bf First Name & \bf Last Name & \bf Birth Date \\ \hline
				1 &	Lauren & Johnson & 1982-03-12 \\
				2 & Alice & Smith & 1979-12-04 \\
				3 & James & White & 1964-09-01 \\
			\end{tabular}
		}
	
		{\tiny
			\begin{tabular}{lll}
				\bf Patient ID & \bf Drug & \bf Prescription date \\ \hline
				1 & Acetaminophen & 2010-01-06 \\
				2 & Amoxicillin & 1998-01-29 \\
				2 & Lisinopril & 2019-05-26 \\
				3 & Albuterol & 1990-12-15 \\
			\end{tabular}
		}
	
		%TODO: Add other tables
	\end{frame}

	% Then add another drug
	% Then consider adding columns
	% Then add rows
	% Then split up into multiple tables

	% Something about multidimensionality?

	\begin{frame}{Schemas}
		Star schema
		Other schemas?
	\end{frame}

	\begin{frame}{SQL}
		select/insert/delete (+where)
		merge
		group by
		summary functions
	\end{frame}

\end{document}